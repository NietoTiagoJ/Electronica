\documentclass{article}
\usepackage{listings}
\usepackage{xcolor} % Para colores en el código fuente

% Configuración del paquete listings para Verilog
\lstdefinestyle{verilog}{
    language=Verilog,
    basicstyle=\ttfamily\small,
    keywordstyle=\color{blue},
    commentstyle=\color{green},
    stringstyle=\color{red},
    numberstyle=\tiny\color{gray},
    stepnumber=1,
    numbersep=5pt,
    backgroundcolor=\color{lightgray},
    frame=single,
    rulecolor=\color{black},
    showstringspaces=false,
    breaklines=true,           % Permite el ajuste automático de líneas largas
    postbreak=\mbox{\textcolor{red}{$\hookrightarrow$}\space} % Símbolo de continuación
}

\begin{document}

\title{Inclusion of Verilog Code in LaTeX}
\author{Your Name}
\date{\today}
\maketitle

\section{Introduction}
In this document, we include and format Verilog code from two files using the \texttt{listings} package.

\section{Verilog Code from mx.v}

Below is the content of the file \texttt{mx.v}:

\lstinputlisting[style=verilog,

